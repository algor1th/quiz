\chapter{Introduction and project overview}

An important part of modern cybersecurity research is the human factor in IT systems.
Few years ago many companies started trying to minimize the vulnerability of human failure as a core vulnerability for most of their security critical systems. For this purpose they educated their employees with often expensive seminars and professional cybersecurity experts.\\

With this project we try to research a different technique of educating employees by developing an educational cybersecurity game-infrastructure. 
By making the system userfriendly and playful we hope to motivate users to learn cybersecurity subjects in order to success in the competitive game environment. We think that such an approach could be more cost effective and could archieve better results then cybersecurity seminars. Also both approaches can be combined to deepen knowledge, which could be interesting for eg. cybersecurity students. 

Our infrastructure consists of a central user server, which manages the registered users and their scores. As a first example application it also contains a quiz application, where users can challenge each other to compete in answering cybersecurity related questions for gaining score points. 

Further possible applications could be a capture the flag like games or for example a "phishing" application, that test employees behaviour by sending phisihing mails to them. It then could track their reaction and give/take score points of them.\\

The handing over state coveres the userserver and a first version of the quizserver.
In agreement to Oliver Schranz some features of the quizserver are not fully developed and are replaced with placeholder systems. The quizserver should act as a first minimal working product, to research its efficiency and propose a first design draft.